\chapter{Zusammenfassung und Ausblick}
Lorem ipsum dolor sit amet, consetetur sadipscing
elitr, sed diam nonumy eirmod tempor invidunt ut labore
et dolore magna aliquyam erat, sed diam voluptua. At vero eos et
accusam et justo duo dolores et ea rebum. Stet clita kasd gubergren, 
no sea takimata sanctus est Lorem ipsum dolor sit amet. Lorem ipsum dolor 
sit amet, consetetur sadipscing elitr, sed diam nonumy eirmod tempor 
invidunt ut labore et dolore magna aliquyam erat, sed diam voluptua. 
At vero eos et accusam et justo duo dolores et ea rebum. Stet clita kasd 
gubergren, no sea takimata sanctus est Lorem ipsum dolor sit amet.

\section{Zusammenfassung}

Das Ziel dieser Arbeit war die Verbesserung der präklinischen Notfallversorgung, insbesondere die Bewältigung der hohen Ressourcenknappheit und die Optimierung der Patientenversorgung. Darüber hinaus sollte die Arbeit auch die Behandlung in der Klinik verbessern, indem eine Software entwickelt wird, die es auch einem Arzt in der Klinik ermöglicht, Patienten präklinisch zu behandeln. Zur Bestätigung der Hypothesen wurde im Rahmen dieser Arbeit eine Telenotarzt-Software entwickelt.

Zunächst wurden alle funktionalen und nicht-funktionalen Anforderungen für den Entwurf der Software gesammelt. Hierbei wurden insbesondere Anforderungen an die Skalierbarkeit und eine intelligente Möglichkeit zur Benutzeranmeldung über verschiedene Systeme herausgestellt. Die Anforderungen wurden priorisiert und eingegrenzt.

In der Entwurfsphase wurden die Anforderungen zu einem detaillierten Softwareentwurf ausgearbeitet. Dabei wurden wichtige konzeptionelle Designentscheidungen im Detail untersucht und Einblicke in die gesamte Struktur der Software gewährt.

Anschließend erfolgte die Implementierung, bei der konkrete Verfahren zur Umsetzung der Software aufgezeigt wurden. Es wurden Code-Beispiele für bestimmte Stellen hervorgehoben und ein Verständnis für die Interoperabilität des Systems geschaffen.

In der Evaluationsphase wurde analysiert, ob die zuvor vereinbarten Anforderungen eingehalten werden konnten oder ob es Anforderungen gab, die nicht erfüllt werden konnten. Abschließend wurde die Software in einer Einsatzsimulation getestet und die Ergebnisse ausgewertet.

\section{Ausblick}

Die entwickelte Telenotarzt-Software hat bereits beeindruckende Fortschritte in der Verbesserung der präklinischen Notfallversorgung gezeigt. Diese Arbeit markiert jedoch erst den Beginn einer spannenden Entwicklung, die das Potenzial hat, die Art und Weise, wie medizinische Notfälle behandelt werden, grundlegend zu verändern. Im Weiteren werden vielversprechende Aspekte und zukünftige Entwicklungen in diesem Bereich skizziert.

Die Telenotarzt-Software kann in verschiedenen medizinischen Fachbereichen eingesetzt werden, um spezialisierte Notfallversorgung zu bieten. Dies könnte bedeuten, Experten aus Bereichen wie Neurologie, Kardiologie oder Kinderheilkunde einzubeziehen, um die präklinische Diagnose und Behandlung für spezifische Patientengruppen zu optimieren. Ebenso kann die Integration von Künstlicher Intelligenz zur Unterstützung von Telenotärzten die Effizienz und Genauigkeit der Diagnosen und Behandlungspläne weiter steigern. KI kann in Echtzeit Daten analysieren und medizinisches Fachwissen zur Verfügung stellen, um schnelle Entscheidungsfindungen zu unterstützen.

Die kontinuierliche Überwachung und Qualitätssicherung der Telenotarzt-Software sowie die Schulung der beteiligten Fachkräfte sind von entscheidender Bedeutung. Dies schließt auch die Einbeziehung von Datenschutz- und Sicherheitsprotokollen ein, um den Schutz sensibler Gesundheitsdaten zu gewährleisten. Die zukünftige Zusammenarbeit mit verschiedenen Gesundheitssystemen, Krankenhäusern und Notfalldiensten erfordert die Schaffung von branchenübergreifenden Standards und die Gewährleistung der nahtlosen Interoperabilität zwischen verschiedenen Systemen und Plattformen.

Schließlich müssen ethische und rechtliche Aspekte berücksichtigt werden. Die Nutzung von Telenotarzt-Software wirft ethische und rechtliche Fragen auf, die weiter erforscht und adressiert werden müssen. Dies beinhaltet Fragen der Haftung, des Datenschutzes und der Einwilligung der Patienten.

Die Telenotarzt-Software hat das Potenzial, die Notfallmedizin zu revolutionieren und die Versorgung von Patienten in kritischen Situationen zu verbessern. Diese Arbeit hat wichtige Grundlagen gelegt, aber es bleibt noch viel zu tun, um diese Technologie weiterzuentwickeln und sicherzustellen, dass sie effektiv und verantwortungsbewusst eingesetzt wird. Die Zukunft der Telenotarzt-Software verspricht aufregende Entwicklungen, die das Gesundheitswesen nachhaltig beeinflussen werden.


