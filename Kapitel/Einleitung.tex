\chapter{Einleitung}


\section{Motivation}
Die vorliegende Masterarbeit strebt die Optimierung des Informationsaustauschs zwischen Notfallsanitätern und Notärzten an, um eine erstklassige Patientenversorgung zu gewährleisten und den beratenden Arzt bestmöglich zu unterstützen. Das zentrale Ziel besteht darin, eine Software zu entwickeln, die es ermöglicht, den Arzt virtuell am Einsatzgeschehen teilhaben zu lassen. In einem digitalen Paradigmenwechsel soll das bisherige Konzept, bestehend aus einem \ac{NEF} und einem \ac{RTW}, transformiert werden. Diese Digitalisierung ermöglicht die Alarmierung eines virtuellen NEF für den Einsatz. Der Notarzt auf dem virtuellen NEF erhält remote Zugriff auf sämtliche Einsatzdaten, indem eine browserbasierte Anwendung entwickelt wird. Diese Anwendung gewährt dem Notarzt Zugriff auf die digitale Einsatzdokumentation, den Defibrillator, Beatmungsgeräte und Leitstellendaten. Darüber hinaus wird eine WebRTC-Verbindung zu den fest installierten Kameras im Rettungswagen und den mobilen Geräten der Notfallsanitäter hergestellt. Diese bahnbrechende Initiative verspricht eine effizientere Koordination und Kommunikation im Rettungswesen, um letztendlich die Qualität der medizinischen Notfallversorgung zu verbessern.


\section{Aufbau der Arbeit}

In den kommenden Kapiteln werden die grundlegenden Aspekte dieser Arbeit ausführlich erläutert. Dabei werden sowohl fachliche Grundlagen, auf die die Arbeit aufbaut, als auch technische Grundlagen, auf die die Arbeit aufsetzt, behandelt. \\

In den fachlichen Grundlagen wird detailliert auf den Ablauf des Einsatzgeschehens eingegangen. Ziel ist es, dem Leser einen umfassenden Überblick über die Funktionsweise der präklinischen Patientenbehandlung zu vermitteln. Hierbei wird im Besonderen aufgezeigt, wie die einzelnen Schritte des Einsatzablaufs ineinandergreifen, um eine bestmögliche Versorgung der Patienten sicherzustellen. \\

Zudem wird auf die technische Ausstattung am Einsatzort, am Fahrzeug sowie die Ausrüstung des Rettungssanitäters/Notfallsanitäters eingehend erläutert. Dies dient dazu, dem Leser einen umfassenden Einblick zu verschaffen, um nachvollziehen zu können, auf welchen Geräten welche Informationen verfügbar sind.  \\


In den technischen Grundlagen werden hingegen die verwendeten Technologien betrachtet. Dabei wird die Software in ihre einzelnen Komponenten aufgebrochen und jede Komponente anschließend detailliert beleuchtet. \\

Es erfolgt eine ausführliche Erläuterung der mobilen Endgeräte am Einsatzort sowie der darauf verwendeten Softwaretechnologien, die die Anwendung ermöglichen. Ebenso werden die erforderlichen Schnittstellentechnologien zu Drittsystemen beschrieben, um Zugriff auf einsatztaktische Daten zu erhalten.\\

Die \ac{TNA-Z} und ihr Aufbau, einschließlich der eingesetzten Softwaretechnologien, werden eingehend erläutert. Abschließend wird in den Grundlagen ausführlich über die eingesetzte Software im Backend berichtet, einschließlich der Struktur und des Aufbaus der Infrastruktur. Dabei wird unter anderem aufgezeigt, wie mit Hilfe von Microservices und Kubernetes eine hohe Verfügbarkeit erreicht wird.\\

In den nachfolgenden Kapiteln werden Lorem ipsum dolor sit amet, consetetur sadipscing elitr, sed diam nonumy eirmod tempor invidunt ut labore et dolore magna aliquyam erat, sed diam voluptua. At vero eos et accusam et justo duo dolores et ea rebum. Stet clita kasd gubergren, no sea takimata sanctus est Lorem ipsum dolor sit amet. Lorem ipsum dolor sit amet, consetetur sadipscing elitr, sed diam nonumy eirmod tempor invidunt ut labore et dolore magna aliquyam erat, sed diam voluptua. At vero eos et accusam et justo duo dolores et ea rebum. Stet clita kasd gubergren, no sea takimata sanctus est Lorem ipsum dolor sit amet. \\


\section{Ziel der Arbeit}
Das Ziel dieser Studie besteht darin,
den Informationsaustausch zwischen Notfallsanitätern und Notärzten zu optimieren, um eine bestmögliche Patientenversorgung sicherzustellen und den konsultierenden Arzt bestmöglich zu unterstützen. Dies soll durch die Entwicklung einer Software erreicht werden, die es ermöglicht, den Arzt virtuell am Einsatzgeschehen teilhaben zu lassen. In kurzen Worten gesagt, das bestehende Konzept mit einem \ac{NEF} und einem \ac{RTW} soll digitalisiert werden.

Durch diese Digitalisierung wird es möglich, einen virtuellen \ac{NEF} für den Einsatz zu aktivieren. Der Notarzt auf dem virtuellen \ac{NEF} erhält remote Zugriff auf sämtliche Einsatzdaten. Hierfür wird eine browserbasierte Anwendung entwickelt, die dem Notarzt den Zugriff auf die digitale Einsatzdokumentation, den Defibrillator, Beatmungsgeräte und Leitstellendaten ermöglicht. Zusätzlich wird eine WebRTC-Verbindung zu den fest installierten Kameras im Rettungswagen und den mobilen Geräten der Notfallsanitäter hergestellt.
\\

Daraus ergeben sich die folgenden Hypothesen:

\begin{itemize}
    \item \textbf{1 Hypothese}: Der Einsatz einer TNA-Software reduziert den Personalmangel in der präklinischen Notfallversorgung.
    
    \item \textbf{2 Hypothese}: Die präklinische Patientenversorgung wird durch den Einsatz der Software signifikant verbessert.
    
    \item \textbf{3 Hypothese}: Es werden zusätzliche Ressourcen in Kliniken geschaffen.
  \end{itemize}
